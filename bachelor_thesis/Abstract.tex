Many biological processes are driven by interactions between molecules diffusing through the cy\-to\-sol of the cells. Bindings between two molecule types are one sort of interaction. Here, the binding fraction is a quantity of interest. It depends on the strength of the binding and the concentration of the dissolved molecules. The measurement of binding fractions is an indispensable tool for understanding biological processes because it indicates the probability of a molecular interaction. A typical device for such measurements is a confocal fluorescence microscope. In a confocal fluorescence microscope, two types of freely diffusing dye-labeled molecules in a solution are illuminated by a light source, and the emitted fluorescence photons are detected. Different molecule types are distinguished by different dye-labelings. Thus, a molecular complex in the observation volume leads to two photon bursts of different fluorescence colors. Typically, the observation volumes for both colors do neither have the same size nor overlap completely. An approach to determine the binding fraction from such measurements is called \glsfirst{BTCCD}. It applies a brightness-gating, where only photon bursts with a minimal number of photons, called brightness threshold, are selected. Thereby, dim bursts that do not correspond to trajectories through both observation volumes, and that lead to an underestimation of the coincidence fraction are excluded. \\

A coincidence of two photon bursts can also occur if two independent molecules are simultaneously inside the observation volume. Such events are called chance coincidences and lead to an overestimation of the binding fraction. In this thesis, the limitation of \gls{BTCCD} due to the occurrence of chance coincidences is investigated. It is shown that an existing approach to correct for chance coincidences is only applicable in a limited range of sample concentrations that depends on the diffusing properties of the dissolved molecules. Based on experimental data, a method is proposed that allows choosing a suitable sample concentration, for which the chance coincidence correction is applicable. A trial experiment validates the method. \\

Besides the investigation of chance coincidences, other properties of \gls{BTCCD} are analyzed. 

First of all, it is shown that roughly \num{10000} photon bursts need to be recorded if the uncertainty on the coincidence fraction before the correction should be smaller than \SI{5}{\percent}. 

Relevant quantities that occur during the application of \gls{BTCCD} are the average number of molecules inside the observation volume, the dwell time, and the molecular brightness. In this thesis, the dependency of these quantities on the brightness threshold is investigated. Experimental data reveal that their development is guided by two processes that take place on the time scale of the dwell time. A fast process is identified as the exclusion of dim bursts due to the application of brightness-gating. The underlying sources for a second, slower process remain unknown.