The basic unit of every living organism is the biological cell. In the cytosol of the cell, diverse interactions between biomolecules, such as DNA, RNA, and proteins, occur. Biology investigates the connection between those molecular processes and the functionality of life. Physicists develop the tools to measure molecular processes. Thus, the application of physical methods and investigation techniques for purposes of biology is called biophysics. Considering that the measurement of molecular processes is indispensable for the understanding of living organisms, the relevance of biophysics becomes evident \cite{Zaccai2017}.\\

Bindings between two molecule types are one sort of interaction that drives biological processes. Here, the binding fraction is a quantity of interest. It depends on the strength of the binding and the concentration of the dissolved molecules. The binding fraction indicates the probability of a molecular interaction between the two molecule types. Hence, the measurement of binding fractions is essential for an understanding of biological processes \cite{Lakowicz2006}. \\

Fluorescence-based techniques are one method to measure binding fractions. There, the fluorescence response of a molecule after the excitation with light is measured. Most bio\-mol\-e\-cules are not fluorescent in the visible spectrum. In this case, a typical approach is to attach a fluorescent dye to the biomolecule in order to make it observable.

The advantage of fluorescence is its selectivity. Only molecules that carry a fluorescent component can be detected. Therefore, it is only sensitive for the molecules of interest, and background, e.g., of a non-fluorescent solvent, is suppressed. Since fluorescence does not alter the properties of a biomolecule, it is an appropriate technique to examine its biological functionality. Finally, fluorescence allows the investigation of dynamic processes as the fluorophore is a part of the biomolecule and moves with it \cite{Lakowicz2006}. \\

An approach to determine the binding fraction between two molecule types is called \glsfirst{BTCCD}. If the two molecule types are labeled with different dyes, the simultaneous occurrence of two fluorescence colors indicates the presence of a molecular complex inside the observation volume. Typically, the observation volumes for two different wavelengths have a different size and do not overlap completely. Thus, the coincidence fraction is underestimated. \gls{BTCCD} takes this systematic deviation into account and selects only molecular trajectories that lead through both observation volumes. Another phenomenon is the random presence of two different single-labeled molecules inside the observation volume at the same time. These chance coincidences do not result from an actual chemical complex, and should, thus, not increase the binding fraction. It has already been proposed an approach to correct for chance coincidences \cite{Hoefig2020}. 

The application of \gls{BTCCD} requires the detection of single molecules, and, thus, low sample concentrations. A typical criterion is that the average number of molecules per observation volume needs to be less than \num{0.03} to make the probability of multi-molecule events sufficiently small \cite{Gopich2008}. \\

The experimental part of this thesis is divided into three chapters. Chapter~\ref{Chapter:DependenciesOnBrightnessThreshold} is concerned with relevant quantities that occur during the application of \gls{BTCCD}. Mathematical dependencies for those quantities, derived under several theoretical assumptions, are compared with experimental results, and, if necessary, adjusted. In Chapter~\ref{Chapter:OptimalNumberOfBursts}, the required measurement size to obtain the coincidence fraction with an uncertainty of less than \SI{5}{\percent} is investigated. The central part of the thesis is Chapter~\ref{Chapter:ChanceCoincidenceLimitation}. There, the application limits of the existing correction for chance coincidences are examined. Phenomenologically, a method to decide prior to a measurement if the correction is suitable for particular measurement parameters is proposed. A trial measurement validates the method. 
