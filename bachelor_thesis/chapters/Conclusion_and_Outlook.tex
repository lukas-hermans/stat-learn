The central part of this thesis was concerned with the occurrence of chance coincidences during the application of \glsfirst{BTCCD}. The investigation of a sample with an expected binding fraction of \SI{0}{\percent} verified that the chance coincidence fraction increases with the molecule number and the dwell time ratio of the involved samples. \\ 

Chance coincidences lead to a systematic overestimation of the coincidence fraction. In this thesis, the limitation of \gls{BTCCD} due to chance coincidences was investigated. For this purpose, the applicability of an existing approach to correct for chance coincidences was analyzed. It was shown that the chance coincidence correction only leads to accurate results for a limited range of dwell time ratios that depends on the molecule number. The knowledge of the application range allows choosing a suitable molecule number for given samples, for which the correction is applicable. \\

The derived application limits of the chance coincidence correction were validated by a trial experiment, where the expected binding fraction was \SI{50}{\percent}. The validation revealed an existing shortcoming of \gls{BTCCD}. It cannot be applied for sample molecules with significantly differing diffusing properties if the actual binding fraction is larger than \SI{0}{\percent}. A possible solution could be substituting the brightness threshold with another parameter, e.g., the molecular brightness or the \gls{IPL}. However, this situation complicated the validation process because only a limited range of dwell time ratios during the application of \gls{BTCCD} was accessible. Therefore, the corrected coincidence fraction could solely be determined for dwell time ratios slightly outside the application range. However, even those values confirm the derived application limits of the chance coincidence correction. \\

All in all, the occurrence of chance coincidences limits the precision with which the binding fraction can be determined. Thus, prior to the application of \gls{BTCCD}, it has to be checked that the expected range of dwell time ratios lies inside the application range of the chance coincidence correction. In practice, the sample concentration has to be adjusted to the specific diffusing properties of the molecules under investigation. \\

Besides the application limits of chance coincidences, other properties of \gls{BTCCD} were analyzed. 

First of all, it was shown that roughly \num{10000} photon bursts need to be recorded if the uncertainty on the coincidence fraction before the correction should be smaller than \SI{5}{\percent}. 

Moreover, the dependencies of the average values of dwell time, molecule number, and molecular brightness on the brightness threshold were investigated. Based on theoretical assumptions, mathematical expressions for those quantities were derived. The experiment revealed that their development is guided by two processes that take place on the time scale of the dwell time. A fast process is identified as the exclusion of dim bursts due to the application of brightness-gating. Several mechanisms that influence fluorescence characteristics could be the underlying reason for a second, slower process. One of these is the transition of electrons in a triplet state. Another relevant effect could be the charge of the used fluorescent dyes and their Coulomb interaction with the sample, e.g., with the overall positive charge of DNA. The specific reasons for the second process remain unknown and are object of further research. \\

Finally, characterizing the starting parameters of a \gls{BTCCD} analysis, it was noted that the smoothing of the \gls{IPL} time trace leads to nonphysical bursts that contain zero or one photon. For these bursts, no meaningful dwell time can be defined. Thus, \gls{BTCCD} could be improved if those bursts were excluded before the application of a brightness threshold.