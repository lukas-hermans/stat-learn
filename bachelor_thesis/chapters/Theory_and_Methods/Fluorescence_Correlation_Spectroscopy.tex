One method to process the data acquired with the confocal microscope is \glsfirst{FCS}. In contrast to \gls{BTCCD}, it does not require that the probability for more than one molecule inside the confocal detection volume is negligible. Thus, it can also be used for higher concentrated sample solutions.\\

\gls{FCS} is based on the normalized auto-correlation function that is in general defined as
\begin{equation}
	G(\tau) = \frac{\left\langle \delta F(t) \delta F(t + \tau) \right\rangle_t}{\left\langle F(t) \right\rangle_t^2},
\end{equation}
where $\delta F(t) = F(t) - \left\langle F(t) \right\rangle_t$. Here, $F$ is the fluorescence intensity of one channel that depends on the macro time $t$. $\langle \cdot \rangle_t$ denotes the time average. $G$ is a measure for the self-similarity of the fluorescence signal after a lag time $\tau$. Note, that the lag time should not be confused with the micro time defined in the previous section.\\

On the one hand, $G$ can be calculated directly from the measurement data since the detected number of photons per time and, thus, the fluorescence intensity $F$ is known. On the other hand, a theoretical expression for $G$ can be derived. Its specific form depends on the properties of the dissolved sample. If  a three-dimensional Gaussian describes the confocal detection volume, see Equation~\eqref{Equation:3DGaussian}, the normalized auto-correlation function for freely diffusing particles with triplet states of lifetime $\tau_{triplet}$ is given by
\begin{equation} \label{Equation:AutoCorrelationTriplet}
	G(\tau) = G(0) \cdot \left(1 + \frac{T}{1 - T} \cdot e^{-\tau/ \tau_{triplet}} \right) \cdot \frac{1}{1 + \frac{4D\tau}{r_0^2}} \cdot \frac{1}{\sqrt{1 + \frac{4D\tau}{\kappa^2r_0^2}}}.
\end{equation}
Here, $T$ is the average fraction of particles in a triplet state, $D$ the diffusion coefficient of the sample, and $\kappa = z_0/ r_0$ \cite{Schwille2001}.\\ 

\gls{FCS} can be used to determine the effective volume $V_{eff}$, defined by Equation~\eqref{Equation:DefinitionConfocalVolume}. For this purpose, the diffusion coefficient $D$ has to be known. Its value for typical reference samples and its temperature dependency can be found in \cite{Kapusta2010}. Then, the measurement data can be fitted with Equation~\eqref{Equation:AutoCorrelationTriplet}, where the fit parameters are $G(0)$, $T$, $\tau_{triplet}$, $r_0$, and $\kappa$. Finally, the effective volume is \cite{ConfocalVolumeDetermination} 
\begin{equation}
	V_{eff} = \pi^{3/2} \cdot r_0^3 \cdot \kappa.
\end{equation}\\

Once having determined the effective volume for both channels of the setup, \gls{FCS} was mainly applied to identify the concentration $C$ of a sample. To do so, the measurement results were again fitted with Equation~\eqref{Equation:AutoCorrelationTriplet}. Then, the relation
\begin{equation} \label{Equation:ConcreteG}
	G(0) = \frac{1}{C \cdot V_{eff}}
\end{equation}
allows to calculate the sample concentration. In some cases, the average number of molecules $\left\langle N(t) \right\rangle_t$ inside the effective volume instead of the concentration is the quantity of interest. The fluorescence intensity $F$ is proportional to $N$. Taking into account that $N$ is Poisson distributed, yields
\begin{equation} \label{Equation:AbstractG}
	G(0) = \frac{\left\langle \left( \delta F(t) \right)^2 \right\rangle_t}{\left\langle F(t) \right\rangle_t^2} = \frac{\left\langle \left( \delta N(t) \right)^2 \right\rangle_t}{\left\langle N(t) \right\rangle_t^2} = \frac{\left\langle N(t) \right\rangle_t}{\left\langle N(t) \right\rangle_t^2} = \frac{1}{\left\langle N(t) \right\rangle_t}.
\end{equation}
Finally, the comparison of Equations~\eqref{Equation:ConcreteG} and \eqref{Equation:AbstractG} leads to
\begin{equation}
	\left\langle N(t) \right\rangle_t = C \cdot V_{eff}
\end{equation}
so that \gls{FCS} allows the determination of both the sample concentration $C$ as well as the average number of molecules $\left\langle N(t) \right\rangle_t$ inside the confocal detection volume \cite{Schwille2001}.\\ 

Furthermore, a fit of $G$ gives a value for the diffusion coefficient $D$ of the sample, so that the diffusion properties can be checked. There are diverse reasons for possible deviations of the measured diffusion coefficient from an expectation. If the measured value for $D$ is lower, the sample molecules can, for instance, aggregate to larger complexes. If it is higher, the molecules may be partially dissociated into smaller components, or free fluorescent dyes could be present.