\begin{otherlanguage}{ngerman}
Viele biologische Prozesse basieren auf Wechselwirkungen zwischen Molekülen, die frei durch das Cytosol der Zellen diffundieren. Eine Art von Wechselwirkungen sind Bindungen zwischen zwei Molekültypen. Dabei ist das Bindungsverhältnis ein Untersuchungsgegenstand. Es hängt von der Bindungsstärke und der Konzentration der Moleküle ab. Die Messung von Bindungsverhältnissen hat eine hohe Bedeutung für die Untersuchung von biologischen Prozessen, weil es ein Indikator für die Wahrscheinlichkeit von molekularen Wechselwirkungen ist. Ein verbreitetes Messgerät ist das konfokale Fluoreszenzmikroskop. Dabei werden zwei Typen von frei diffundierenden Farbstoff-markierten Molekülen in einer Lösung von einer Lichtquelle beleuchtet und die emittierten Fluoreszenzphotonen detektiert. Die Molekültypen werden durch verschiedene Farbstoffmarkierungen unterschieden. Daher führt die Anwesenheit eines Molekülkomplexes im Beobachtungsvolumen zu zwei koinzidenten \enquote{Photonschwallen} verschiedener Wellenlänge. Typischerweise sind die Beobachtungsvolumen beider Farben nicht gleich groß und zudem relativ zueinander verschoben. Eine Methode zur Bestimmung des Bindungsverhältnisses aus derartigen Messungen wird helligkeitsgeleitete Zweifarben-Ko\-in\-zi\-denz\-de\-tek\-tion (BTCCD) genannt. Dabei wird eine Helligkeitsschwelle angewendet, durch die nur Schwalle mit einer Mindestanzahl von Photonen ausgewählt werden. Dadurch werden schwache Schwalle, die nicht aus zentralen Trajektorien durch beide Beobachtungsvolumen resultieren und zu einem unterschätzten Koinzidenzverhältnis führen, aussortiert. \\

Das Auftreten zweier koinzidenter Photonschwallen kann auch auf die gleichzeitige Präsenz zwei unabhängiger Moleküle zurückzuführen sein. Solche Ereignisse werden Zufallskoinzidenzen genannt. Sie führen zu einem systematisch überschätzten Bindungsverhältnis. In dieser Arbeit wird untersucht, inwiefern Zufallskoinzidenzen die Anwendung von \gls{BTCCD} limitieren. Dabei wird gezeigt, dass ein existierendes Verfahren zur Korrektur von Zufallskoinzidenzen nur in einem beschränkten Bereich im Bezug auf die Probenkonzentration angewendet werden kann. Der Bereich hängt von den Diffusionseigenschaften der Probenmoleküle ab. Basierend auf experimentellen Ergebnissen wird eine Methode entwickelt, die es erlaubt eine geeignete Probenkonzentration zu wählen, für die die Korrektur von Zufallskoinzidenzen verwendbar ist. Die Methode wird durch ein Beispielexperiment validiert. \\

Neben der Untersuchung von Zufallskoinzidenzen werden weitere Eigenschaften von \gls{BTCCD} analysiert. 

Zunächst wird gezeigt, dass etwa \num{10000} Photonschwalle registriert werden müssen, um das unkorrigierte Koinzidenzverhältnis mit einer Unsicherheit von weniger als \SI{5}{\percent} bestimmen zu können.

Relevante Größen bei der Verwendung von \gls{BTCCD} sind neben der mittleren Anzahl der Moleküle im Beobachtungsvolumen die Verweildauer der Moleküle und die molekulare Helligkeit. In der vorliegenden Arbeit wird die Abhängigkeit dieser Größen von der Helligkeitsschwelle untersucht. Experimentell wird festgestellt, dass die Entwicklung der Größen durch zwei Prozesse bestimmt wird, die auf der Zeitskala der Verweildauer stattfinden. Ein schneller Prozess wird auf den Ausschluss schwacher Photonschwalle durch das Anwenden der Helligkeitsschwelle zurückgeführt. Die zugrundeliegenden Ursachen für einen zweiten, langsameren Prozess bleiben unbekannt. 
\end{otherlanguage}
