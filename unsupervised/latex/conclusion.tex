The exemplary social media analysis of Elon Musk's Twitter account proves that text mining techniques and unsupervised clustering techniques are applicable for the profiling of social media users. The retrieved profiles can for example be used to show the users personalized advertisments. The combination of the results of an initial exploratory data examination, and the application of K-means clustering and hierarchical clustering is summarized in Tab.~\ref{tab:musk}. The analyses in the present work lead to a detailed profile of Elon Musk. A comparison with the general knowledge about Elon Musk and his companies reveals the accuracy of the obtained profile. For the analyses in the present work, the K-means clustering is applied with $K=8$ desired clusters, while the hierarchical clustering leads to $15$ interpretable clusters. These differences lie in the different natures of the clustering approaches.