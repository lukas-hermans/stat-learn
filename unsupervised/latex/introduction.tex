Recently, the application of computer vision techniques has gained increasing importance in a broad variety of fields, e.g. for the development of autonomeous vehicles, for facial recognition, and for the detection of illnesses \cite{autonomeous2020, emotion2020, medicine2021}. This process is especially driven by the growing field of machine learning. \\

A common dataset to benchmark and compare machine learning algorithms for image recognition as a subfield of computer vision is the MNIST database of handwritten digits. The dataset contains $60000$ training examples and $10000$ test examples of handwritten digits from $0$ to $9$. Originally, the dataset was published by Y. LeCun et al \cite{MNIST}. In this paper, in order to simplify the work with the images, a revision of the original dataset is adopted \cite{KaggleData}. In the following, the revised dataset is simply refered to as MNIST dataset. \\

The objective of the present work is to train a multiclass kernel perceptron algorithm for the classification of handwritten digits using the training examples in the MNIST dataset. \\

In Section~\ref{sec:dataset_definitions}, the MNIST dataset is presented in more detail, and some basic definitions and notations are clarified. Then, in Section~\ref{sec:theory}, the theoretical foundation of the multiclass kernel perceptron algorithm is summarized. There, three different approaches to retrieve a multiclass predictor from the multiclass kernel perceptron algorithm are distinguished. Section~\ref{sec:implementation_software} describes the details of the implementation of the multiclass kernel perceptron algorithm in the present work and gives a list of the applied software. The results of the training of predictors using the multiclass kernel perceptron algorithm on the training part of the MNIST dataset are presented in Section~\ref{sec:results}. In doing so, the hyperparameters of the multiclass kernel perceptron algorithm are optimized, in order to obtain three multiclass predictors with the lowest error rates on the test part of the MNIST dataset. In Section~\ref{sec:discussion}, the main advantages as well as the drawbacks of the application of the multiclass kernel perceptron algorithm for the application on the MNIST dataset are discussed. The retrieved predictors are compared with the test error rates of different learning algorithms that are applied to the MNIST dataset in the literature. Finally, Section~\ref{sec:conclusion_outlook} summarizes the present work and gives an overview of current and future objects of research regarding the recognition of handwritten digits.