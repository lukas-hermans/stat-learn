\title{\large Unsupervised Learning \\ \LARGE
 Elon Musk:\\An Exemplary Twitter User Analysis}
\author{Lukas Hermans\\ \\
{Università degli Studi di Milano} \\
\href{mailto:lukas.hermans@studenti.unimi.it}
{lukas.hermans@studenti.unimi.it}}

\maketitle

\begin{abstract} 
\noindent
\fontdimen2\font=0.7ex% inter word space
The analysis of social media profiles plays an essential role in a variety of fields, e.g., for companies that seek to increase their returns via targeted advertising. In the present work, an exemplary user analysis of the Twitter profile of the tech entrepreneur Elon Musk is provided. The dataset consists of every Tweet posted by Elon Musk until the end of 2020. His Twitter activity has increased significantly since his first Tweet in 2010. An initial data examination reveals the dominance of his companies Tesla and SpaceX in the Tweets, as well as his interest in economic topics. The main part of the present work consists in the clustering of the Tweets by applying unsupervised machine learning techniques. The application of K-means clustering offers informative insights for $K=10$ clusters. Hierarchical clustering leads to $15$ interpretable clusters. A comparison of the obtained insights with the general knowledge about Elon Musk reveals a precise description of his career, interests,  and personality by the applied clustering techniques. The exemplary analysis of Elon Musk's Twitter profile proofs the applicability of text mining and unsupervised clustering techniques for social media profiling. 
\end{abstract}
